\documentclass[letter, 11pt]{article}

\usepackage{amsmath,amsthm,amssymb}
\usepackage{fancyhdr}
\usepackage{geometry}
\usepackage{enumerate}
\usepackage{enumitem}
\usepackage{listings}
\usepackage{algorithm}
\usepackage{algorithmic}
\usepackage{eqparbox}
\usepackage{float}

\author{Shengjie Li}
\title{Homework 1}

\pagestyle{fancy}
\fancyhf{} 
\lhead{Shengjie Li \\ RUID: 188008047}
\cfoot{\thepage} 
\renewcommand{\headrulewidth}{1pt}
\renewcommand{\headwidth}{\textwidth}
\renewcommand\algorithmiccomment[1]{%
	\hfill\#\ \eqparbox{COMMENT}{#1}%
}

\setlength\parindent{0pt}

% margin adjustment
\addtolength{\textwidth}{1in}
\addtolength{\oddsidemargin}{-0.375in}
\addtolength{\evensidemargin}{-0.375in}
\addtolength{\topmargin}{-.5in}
\addtolength{\textheight}{1.0in}
\setlength\parindent{0cm}

\begin{document}
	\centerline{Homework 1}
	\begin{enumerate}[wide = 0pt, label = \textbf{Problem \arabic*:}]
		\item {Consider the matrix \[ A = \begin{bmatrix}
			1 & 0.5 \\
			0 & 1 + \epsilon
			\end{bmatrix}. \]} 
		\begin{enumerate}
			\item {Find the eigenvalues/eigenvectors of $ A $ assuming $ \epsilon \ne 0 $. Force your eigenvectors to have unit norm.} 
			\begin{align*}
				A - \lambda I &= \begin{bmatrix}
								1 - \lambda & 0.5 \\
								0 & 1 + \epsilon - \lambda
								\end{bmatrix} \\
				\det(A - \lambda I) &= (1 - \lambda) * (1 + \epsilon - \lambda) - 0.5 * 0 \\
				&= (1 - \lambda) * (1 + \epsilon - \lambda)
			\end{align*}
			Let $ \det(A - \lambda I) = 0 $ we can get eigenvalues $ \lambda_1 = 1, \lambda_2 = 1 + \epsilon $. \\
			For $ \lambda = 1 $, solve $ A \vec{x} = \lambda \vec{x} $: 
			\begin{align*}
				\begin{bmatrix}
				1 & 0.5 \\
				0 & 1 + \epsilon
				\end{bmatrix}
				\begin{bmatrix}
				x_1 \\ x_2
				\end{bmatrix}
				&= 
				\begin{bmatrix}
				x_1 \\ x_2
				\end{bmatrix}
				\\ \downarrow \\
				x_1 + 0.5x_2 &= x_1 \\
				(1 + \epsilon) x_2 &= x_2 
				\\ \downarrow \\
				x_1 &= x_1 (x_1 \ne 0)\\
				x_2 &= 0
			\end{align*}
			Therefore, $ \begin{bmatrix} 1 \\ 0 \end{bmatrix} $ is a eigenvector of $ A $ associated with the eigenvalue $ \lambda = 1 $. 
			\begin{align*}
				\begin{bmatrix}
				1 & 0.5 \\
				0 & 1 + \epsilon
				\end{bmatrix}
				\begin{bmatrix}
				x_1 \\ x_2
				\end{bmatrix}
				&= 
				(1 + \epsilon)
				\begin{bmatrix}
				x_1 \\ x_2
				\end{bmatrix}
				\\ \downarrow \\
				x_1 + 0.5x_2 &= (1 + \epsilon) x_1 \\
				(1 + \epsilon) x_2 &= (1 + \epsilon) x_2 
				\\ \downarrow \\
				x_1 &= \frac{x_2}{2 \epsilon}\\
				x_2 &= x_2
			\end{align*}
			Therefore, $ \begin{bmatrix} \frac{1}{\sqrt{1 + 4 \epsilon^2}} \\ \frac{2 \epsilon}{\sqrt{1 + 4 \epsilon^2}} \end{bmatrix} $ is a eigenvector of $ A $ associated with the eigenvalue $ \lambda = 1 + \epsilon $. \\
			
			\item {Diagonalize $ A $ using the eigenvalues/eigenvectors you computed.} \\
			Let $ T $ be the matrix with eigenvectors as its columns.
			\begin{align*}
				T &= 
				\begin{bmatrix}
				1 & \frac{1}{\sqrt{1 + 4 \epsilon^2}} \\
				0 & \frac{2 \epsilon}{\sqrt{1 + 4 \epsilon^2}}
				\end{bmatrix}
				\\
				T^{-1} &= \frac{1}{\frac{2 \epsilon}{\sqrt{1 + 4 \epsilon^2}}} 
				\begin{bmatrix}
				\frac{2 \epsilon}{\sqrt{1 + 4 \epsilon^2}} & -\frac{1}{\sqrt{1 + 4 \epsilon^2}} \\
				0 & 1
				\end{bmatrix}
				\\
				&=
				\begin{bmatrix}
				1 & -\frac{1}{2 \epsilon} \\
				0 & \frac{\sqrt{1 + 4 \epsilon^2}}{2 \epsilon}
				\end{bmatrix}
				\\
				T^{-1}AT &= 
				\begin{bmatrix}
				1 & -\frac{1}{2 \epsilon} \\
				0 & \frac{\sqrt{1 + 4 \epsilon^2}}{2 \epsilon}
				\end{bmatrix}
				\begin{bmatrix}
				1 & 0.5 \\
				0 & 1 + \epsilon
				\end{bmatrix}
				\begin{bmatrix}
				1 & \frac{1}{\sqrt{1 + 4 \epsilon^2}} \\
				0 & \frac{2 \epsilon}{\sqrt{1 + 4 \epsilon^2}}
				\end{bmatrix}
				\\
				&=
				\begin{bmatrix}
				1 & -\frac{1}{2 \epsilon}\\
				0 & \frac{(1 + \epsilon)\sqrt{1 + 4 \epsilon^2}}{2 \epsilon}
				\end{bmatrix}
				\begin{bmatrix}
				1 & \frac{1}{\sqrt{1 + 4 \epsilon^2}} \\
				0 & \frac{2 \epsilon}{\sqrt{1 + 4 \epsilon^2}}
				\end{bmatrix}
				\\
				&=
				\begin{bmatrix}
				1 & 0 \\
				0 & 1 + \epsilon
				\end{bmatrix}
			\end{align*} \\
			
			\item {Start now sending $ \epsilon \to 0 $. What do
				you observe is happening to the matrices you use for diagonalization as $ \epsilon $ becomes smaller and smaller? So
				what do you conclude when $ \epsilon = 0 $?} \\
			\\
			When $ \epsilon $ is becoming closer to 0, the determinant of matrix $ T $ is becoming closer to 0. \\
			Thus, when $ \epsilon = 0 $, matrix $ T $ will become non-invertable. Matrix $ A $ will become non-diagonalizable.
		\end{enumerate}
	\end{enumerate}
\end{document}
